\chapter{Kurzfassung}

Die Verf�gbarkeit digitaler Informationen nimmt stetig zu. Um die daraus resultierenden komplexen Analyseaufgaben zu bewerkstelligen, wird die Anwendung von Informationsvisualisierung dabei immer wichtiger. Informationsvisualisierung erlaubt uns Muster in gro�en Datenbest�nden zu erkennen und Fakten offenzulegen, die bei der Betrachtung der Rohdaten keinesfalls ersichtlich sind. Die daraus gewonnenen Erkenntnisse k�nnen weitere Nachforschungen veranlassen und schlussendlich als Grundlage f�r wichtige strategische Entscheidungen dienen.

Diese Arbeit befasst sich mit der Erstellung interaktiver Visualisierungen f�r web-basierte Systeme. Visuelle Datenanalyse wird vorwiegend durch spezialisierte Software betrieben, die in lokalen Umgebungen (am Desktop) installiert ist. Diese Systeme greifen auf Dateien und Datenbanken zur�ck um Informationen einzulesen.

Die neuesten Entwicklungen im Bereich der Internet Technologien bieten nun jedoch die Basis f�r die Erstellung interaktiver Visualisierungen in einer verteilten Umgebung, dem Web. Aufgaben, die bisher lokal bewerkstelligt werden mussten und manuellen Datenaustausch erforderten, k�nnen nun durch die Verwendung von kollaborativen Web-Services gel�st werden. Die Tatsache, dass die Installation von Software entf�llt erm�glicht es einer wesentlich breiteren Anwenderschicht visuelle Analysewerkzeuge mit Hilfe ihres Web-Browers zu verwenden.

Diese Arbeit ist an Entwickler von Visualisierungen gerichtet und konzentriert sich auf den Prozess der Erstellung web-basierter interaktiver Visualisierungen. Zun�chst werden, basierend auf dem aktuellen Forschungs- und Technologiestand, existierende Methoden und Technologien vorgestellt und deren Unterschiede hervorgehoben. Basierend auf HTML5 Canvas, einem neuen Technologiestandard zum Rendern von Bitmap-Grafiken, erfolgt die Entwicklung von Unveil.js, einem deklarativen und datengetriebenen Visualierungsframework.

Die Arbeit geht ausserdem auf jene Techniken ein, die bei der Entwicklung von Unveil.js zur Anwendung kamen. Abschlie�end erfolgt eine Evaluierung von ausgew�hlten Werkzeugen mit dem Ziel die St�rken und Schw�chen unterschiedlicher Methoden zu verdeutlichen.